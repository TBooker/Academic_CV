%%%%%%%%%%%%%%%%%%%%%%%%%%%%%%%%%%%%%%%%%%%%%%%%%%%%%%%%%%%%%%%%%%%%%%%%
%%%%%%%%%%%%%%%%%%%%%% Simple LaTeX CV Template %%%%%%%%%%%%%%%%%%%%%%%%
%%%%%%%%%%%%%%%%%%%%%%%%%%%%%%%%%%%%%%%%%%%%%%%%%%%%%%%%%%%%%%%%%%%%%%%%

%%%%%%%%%%%%%%%%%%%%%%%%%%%%%%%%%%%%%%%%%%%%%%%%%%%%%%%%%%%%%%%%%%%%%%%%
%% NOTE: If you find that it says                                     %%
%%                                                                    %%
%%                           1 of ??                                  %%
%%                                                                    %%
%% at the bottom of your first page, this means that the AUX file     %%
%% was not available when you ran LaTeX on this source. Simply RERUN  %%
%% LaTeX to get the ``??'' replaced with the number of the last page  %%
%% of the document. The AUX file will be generated on the first run   %%
%% of LaTeX and used on the second run to fill in all of the          %%
%% references.                                                        %%
%%%%%%%%%%%%%%%%%%%%%%%%%%%%%%%%%%%%%%%%%%%%%%%%%%%%%%%%%%%%%%%%%%%%%%%%

%%%%%%%%%%%%%%%%%%%%%%%%%%%% Document Setup %%%%%%%%%%%%%%%%%%%%%%%%%%%%

% Don't like 10pt? Try 11pt or 12pt
\documentclass[10pt]{article}

% The automated optical recognition software used to digitize resume
% information works best with fonts that do not have serifs. This
% command uses a sans serif font throughout. Uncomment both lines (or at
% least the second) to restore a Roman font (i.e., a font with serifs).
%\usepackage{times}
%\renewcommand{\familydefault}{\sfdefault}

% This is a helpful package that puts math inside length specifications
\usepackage{calc}
\usepackage{comment}

% Simpler bibsection for CV sections
% (thanks to natbib for inspiration)
\makeatletter
\newlength{\bibhang}
\setlength{\bibhang}{1em} %1em}
\newlength{\bibsep}
 {\@listi \global\bibsep\itemsep \global\advance\bibsep by\parsep}
\newenvironment{bibsection}%
        {\begin{enumerate}{}{%
%        {\begin{list}{}{%
       \setlength{\leftmargin}{\bibhang}%
       \setlength{\itemindent}{-\leftmargin}%
       \setlength{\itemsep}{\bibsep}%
       \setlength{\parsep}{\z@}%
        \setlength{\partopsep}{0pt}%
        \setlength{\topsep}{0pt}}}
        {\end{enumerate}\vspace{-.6\baselineskip}}
%        {\end{list}\vspace{-.6\baselineskip}}
\makeatother

% Layout: Puts the section titles on left side of page
\reversemarginpar

%
%         PAPER SIZE, PAGE NUMBER, AND DOCUMENT LAYOUT NOTES:
%
% The next \usepackage line changes the layout for CV style section
% headings as marginal notes. It also sets up the paper size as either
% letter or A4. By default, letter was used. If A4 paper is desired,
% comment out the letterpaper lines and uncomment the a4paper lines.
%
% As you can see, the margin widths and section title widths can be
% easily adjusted.
%
% ALSO: Notice that the includefoot option can be commented OUT in order
% to put the PAGE NUMBER *IN* the bottom margin. This will make the
% effective text area larger.
%
% IF YOU WISH TO REMOVE THE ``of LASTPAGE'' next to each page number,
% see the note about the +LP and -LP lines below. Comment out the +LP
% and uncomment the -LP.
%
% IF YOU WISH TO REMOVE PAGE NUMBERS, be sure that the includefoot line
% is uncommented and ALSO uncomment the \pagestyle{empty} a few lines
% below.
%

%% Use these lines for letter-sized paper
\usepackage[paper=letterpaper,
            %includefoot, % Uncomment to put page number above margin
            marginparwidth=1.2in,     % Length of section titles
            marginparsep=.05in,       % Space between titles and text
            margin=0.6in,               % 1 inch margins
            includemp]{geometry}

%% Use these lines for A4-sized paper
%\usepackage[paper=a4paper,
%            %includefoot, % Uncomment to put page number above margin
%            marginparwidth=30.5mm,    % Length of section titles
%            marginparsep=1.5mm,       % Space between titles and text
%            margin=25mm,              % 25mm margins
%            includemp]{geometry}

%% More layout: Get rid of indenting throughout entire document
\setlength{\parindent}{0in}

\usepackage[shortlabels]{enumitem}

%% Reference the last page in the page number
%
% NOTE: comment the +LP line and uncomment the -LP line to have page
%       numbers without the ``of ##'' last page reference)
%
% NOTE: uncomment the \pagestyle{empty} line to get rid of all page
%       numbers (make sure includefoot is commented out above)
%
\usepackage{fancyhdr,lastpage}
\pagestyle{fancy}
\pagestyle{empty}      % Uncomment this to get rid of page numbers
\fancyhf{}\renewcommand{\headrulewidth}{0pt}
\fancyfootoffset{\marginparsep+\marginparwidth}
\newlength{\footpageshift}
\setlength{\footpageshift}
          {0.5\textwidth+0.5\marginparsep+0.5\marginparwidth-2in}
\lfoot{\hspace{\footpageshift}%
       \parbox{4in}{\, \hfill %
                    \arabic{page} of \protect\pageref*{LastPage} % +LP
%                    \arabic{page}                               % -LP
                    \hfill \,}}

% Finally, give us PDF bookmarks
\usepackage{color,hyperref}
\definecolor{darkblue}{rgb}{0.0,0.0,0.3}
\hypersetup{colorlinks,breaklinks,
            linkcolor=darkblue,urlcolor=darkblue,
            anchorcolor=darkblue,citecolor=darkblue}

%%%%%%%%%%%%%%%%%%%%%%%% End Document Setup %%%%%%%%%%%%%%%%%%%%%%%%%%%%


%%%%%%%%%%%%%%%%%%%%%%%%%%% Helper Commands %%%%%%%%%%%%%%%%%%%%%%%%%%%%

% The title (name) with a horizontal rule under it
% (optional argument typesets an object right-justified across from name
%  as well)
%
% Usage: \makeheading{name}
%        OR
%        \makeheading[right_object]{name}
%
% Place at top of document. It should be the first thing.
% If ``right_object'' is provided in the square-braced optional
% argument, it will be right justified on the same line as ``name'' at
% the top of the CV. For example:
%
%       \makeheading[\emph{Curriculum vitae}]{Your Name}
%
% will put an emphasized ``Curriculum vitae'' at the top of the document
% as a title. Likewise, a picture could be included:
%
%   \makeheading[\includegraphics[height=1.5in]{my_picutre}]{Your Name}
%
% the picture will be flush right across from the name.
\newcommand{\makeheading}[2][]%
        {\hspace*{-\marginparsep minus \marginparwidth}%
         \begin{minipage}[t]{\textwidth+\marginparwidth+\marginparsep}%
             {\large \bfseries #2 \hfill #1}\\[-0.15\baselineskip]%
                 \rule{\columnwidth}{1pt}%
         \end{minipage}}

% The section headings
%
% Usage: \section{section name}
\renewcommand{\section}[1]{\pagebreak[3]%
    \hyphenpenalty=10000%
    \vspace{1.3\baselineskip}%
    \phantomsection\addcontentsline{toc}{section}{#1}%
    \noindent\llap{\scshape\smash{\parbox[t]{\marginparwidth}{\raggedright #1}}}%
    \vspace{-\baselineskip}\par}

% An itemize-style list with lots of space between items
\newenvironment{outerlist}[1][\enskip\textbullet]%
        {\begin{itemize}[#1,leftmargin=*]}{\end{itemize}%
         \vspace{-.6\baselineskip}}

% An environment IDENTICAL to outerlist that has better pre-list spacing
% when used as the first thing in a \section

\newenvironment{lonelist}[1][\enskip\textbullet]%
        {\begin{list}{#1}{%
        \setlength{\partopsep}{0pt}%
        \setlength{\topsep}{0pt}}}
        {\end{list}\vspace{-.5\baselineskip}}

% An itemize-style list with little space between items
\newenvironment{innerlist}[1][\enskip\textbullet]%
        {\begin{itemize}[#1,leftmargin=*,parsep=0pt,itemsep=0pt,topsep=0pt,partopsep=0pt]}
        {\end{itemize}}

% An environment IDENTICAL to innerlist that has better pre-list spacing
% when used as the first thing in a \section
\newenvironment{loneinnerlist}[1][\enskip\textbullet]%
        {\begin{itemize}[#1,leftmargin=*,parsep=0pt,itemsep=0pt,topsep=0pt,partopsep=0pt]}
        {\end{itemize}\vspace{-.5\baselineskip}}

% To add some paragraph space between lines.
% This also tells LaTeX to preferably break a page on one of these gaps
% if there is a needed pagebreak nearby.
\newcommand{\blankline}{\quad\pagebreak[2]}
\newcommand{\halfblankline}{\quad\vspace{-0.3\baselineskip}\pagebreak[2]}

% Uses hyperref to link DOI
\newcommand\doilink[1]{\href{http://dx.doi.org/#1}{#1}}
\newcommand\doi[1]{doi:\doilink{#1}}

% For \url{SOME_URL}, links SOME_URL to the url SOME_URL
\providecommand*\url[1]{\href{#1}{#1}}
% Same as above, but pretty-prints SOME_URL in teletype fixed-width font
\renewcommand*\url[1]{\href{#1}{\texttt{#1}}}

% For \email{ADDRESS}, links ADDRESS to the url mailto:ADDRESS
\providecommand*\email[1]{\href{mailto:#1}{#1}}
% Same as above, but pretty-prints ADDRESS in teletype fixed-width font
%\renewcommand*\email[1]{\href{mailto:#1}{\texttt{#1}}}

%\providecommand\BibTeX{{\rm B\kern-.05em{\sc i\kern-.025em b}\kern-.08em
%    T\kern-.1667em\lower.7ex\hbox{E}\kern-.125emX}}
%\providecommand\BibTeX{{\rm B\kern-.05em{\sc i\kern-.025em b}\kern-.08em
%    \TeX}}
\providecommand\BibTeX{{B\kern-.05em{\sc i\kern-.025em b}\kern-.08em
    \TeX}}
\providecommand\Matlab{\textsc{Matlab}}

%%%%%%%%%%%%%%%%%%%%%%%% End Helper Commands %%%%%%%%%%%%%%%%%%%%%%%%%%%

%%%%%%%%%%%%%%%%%%%%%%%%% Begin CV Document %%%%%%%%%%%%%%%%%%%%%%%%%%%%

\begin{document}
\makeheading{Tom R. Booker}

\section{Contact Information}

% NOTE: Mind where the & separators and \\ breaks are in the following
%       table.
%
% ALSO: \rcollength is the width of the right column of the table
%       (adjust it to your liking; default is 1.85in).
%
\newlength{\rcollength}\setlength{\rcollength}{1.4in}%
%
\begin{tabular}[t]{@{}p{\textwidth-\rcollength}p{\rcollength}}
t.r.booker@sms.ed.ac.uk & Tel. +447858896621 \\\\
Institute of Evolutionary Biology\\ Ashworth Laboratories\\ Edinburgh, EH9 3FL
\end{tabular}



%\section{Objective}

%Insert text here if you want to
%\begin{innerlist}
%\item More information and auxiliary documents can be found at\\\url{http://www.tedpavlic.com/facjobsearch/}
%\end{innerlist}

\section{Research Interests}

Theoretical and empirical population genetics, evolution, genomics, bioinformatics, statistical analysis


\section{Education}

{\textbf{University of Edinburgh}},
Edinburgh, Scotland
\begin{outerlist}

\item[] Ph.D.,
             \href{http://www.ed.ac.uk/biology/evolutionary-biology} {Evolutionary Genetics},
             \emph{Expected:} Autumn 2018
        \begin{innerlist}
        \item[]- Thesis Title: \emph{Understanding patterns of genetic diversity in the house mouse genome}
        \item[]- Supervisors: Professor Peter Keightley and Professor Brian Charlesworth
        \item[]- As part of my EASTBIO scholarship, I spent 3 months learning bioinformatic approaches and techniques at Fusion Genomics, Vancouver, Canada
        \end{innerlist}

\item[] MSc.,
        \href{http://qgen.bio.ed.ac.uk/}
             {Evolutionary Genetics},
             2013 - 2014 (Distinction)
        \begin{innerlist}
        \item[]- Thesis Title: \emph{Searching for balancing selection on a mimicry supergene in the Batesian mimic} Papilio polytes
        \item[]- Supervisors: Professor Deborah Charlesworth and Dr Rob W. Ness
        \item[]- In addition to the thesis component, the MSc program consisted of courses on statistical analysis, population genetics, molecular evolution and phylogenetics, as well as linkage analysis.
        \end{innerlist}
\end{outerlist}
\vspace{.1in}
\href{}{\textbf{University of Stirling}},
Stirling, Scotland
\begin{outerlist}
\item[] BSc Hons,
        \href{http://www.stir.ac.uk/natural-sciences/about-us/bes/}
             {Ecology}, 2009 - 2013 (First Class)
        \begin{innerlist}
        \item[]- Dissertation Title: \emph{An investigation into the fitness and distribution of a newly discovered allopolyploid species,} Mimulus peregrinus
        \item[]- Supervisor:
                   Dr Mario Vallejo-Marin
        \item[]- I undertook a Summer research project on \emph{Mimulus spp.} in Southern and Central Scotland in 2012 with Dr Vallejo-Marin. Courses taken on ecology, conservation biology and genetics. Spent a year on exchange at Simon Fraser University, Vancouver, Canada.
        \end{innerlist}

\end{outerlist}

\section{Experience \& Skills}
\begin{outerlist}

	\item[] Population genetics
		\begin{innerlist}
			\item[] Including: theory, simulations (both forward-time and coalescent), genome scans, demographic analyses, detecting natural selection
		\end{innerlist}

	\item[] Bioinformatics: 
			\begin{innerlist}
			\item[] Including: Handling high-throughput sequence data, read-mapping, variant calling, \emph{de novo} assembly \\ Attended ``\emph{GATK Best practices for variant discovery}'', Edinburgh, UK (2015).
			\end{innerlist}
		\item[] Statistical Analysis: 
			\begin{innerlist}
			\item[] Including: Linear and non-linear regression, parametric and non-parametric statistics, maximium likelihood estimation.
			\end{innerlist}
		
		
	\item[] Computer skills
		\begin{innerlist}
		\item[] Scripting: Highly competent in Python, R and Bash, experience with C and Perl
		\item[] OS: Ubuntu, Windows, Mac OSX
		\item[] Miscellaneous: Grid Engine clustering systems, git, emacs, ssh/scp, tmux, Microsoft Office
		\end{innerlist}
		
	\item[] Science communication: Written and verbal
	
	
\end{outerlist}


\section{Academic Service}
\begin{outerlist}

	\item[] I have reviewed articles for the following journals:
		\begin{innerlist}
			\item[] \emph{Ecology and Evolution}, \emph{Molecular Biology and Evolution}
		\end{innerlist}

\end{outerlist}


\section{Published papers}
\vspace{-.1275in}
\begin{bibsection}

    
   \item {\bf Booker, T. R.}, Jackson, B. C., \& Keightley, P. D. (2017). ``Detecting positive selection in the genome." \emph{BMC Biology}, 15:98. 

	\item {\bf Booker, T. R.}, Ness, R. W., \& Keightley, P. D. (2017). ``The recombination landscape in wild house mice inferred using population genomic data". \emph{Genetics}, 207(1) 297-309

	\item Keightley, P. D., Campos, J. L., {\bf Booker, T. R.}, \& Charlesworth, B. (2016). ``Inferring the frequency spectrum of derived variants to quantify adaptive molecular evolution in protein-coding genes of Drosophila melanogaster." \emph{Genetics}, 203(2), 975-984.
	
	\item {\bf Booker, T.}, Ness, R. W., \& Charlesworth, D. (2015). ``Molecular evolution: breakthroughs and mysteries in Batesian mimicry". \emph{Current Biology}, 25(12), R506-R508.

\end{bibsection}
 
 \section{Papers in preparation}
\vspace{-.1275in}
\begin{bibsection}
 
    \item {\bf Booker, T. R.}, Charlesworth, B. \& Keightley, P. D. (\emph{In preparation}). ``Estimating parameters of strong positive selection from patterns of genetic diversity in house mice"
    
    \item {\bf Booker, T. R.}, \& Keightley, P. D. (\emph{In preparation}). ``Understanding the forces that shape patterns of genetic diversity in the house mouse genome"

\end{bibsection}
 

\section{Selected Presentations}

\emph{Estimating the parameters of selective sweeps from patterns of diversity around functional elements in wild house mice} Mus musculus castaneus (Oral Presentation)\\
{\bf Population Genetics Group 51} , Bristol, UK \hfill January 2018\\[0.01in]

\emph{Selective sweeps and background selection in the genome of wild house mice,} Mus musculus castaneus \\
{\bf ESEB 2017}, Groningen, Netherlands \hfill August 2017\\[0.01in]
{\bf Population Genetics Group 50}, Cambridge, UK \hfill Jan 2017\\[0.01in]

\emph{Hill-Robertson Interference in wild mice,} Mus musculus castaneus (Oral Presentation)\\
{\bf SMBE}, Gold Coast, Australia  \hfill July 2016\\
{\bf Population Genetics Group 49}, Edinburgh, UK \hfill December 2015\\[0.01in]

\emph{Selective sweeps and background selection in the genome of wild house mice,} Mus musculus castaneus (Poster)\\
{\bf SMBE}, Vienna, Austria \hfill July 2015 \\[0.01in]


\emph{Simulating genome evolution in the house mouse: understanding the contribution of Hill-Robertson interference to patterns of genetic diversity} (Oral Presentation)\\
{\bf Quantitative Genomics} London, UK \hfill May 2015 \\[0.01in]

\halfblankline


\section{Academic Honours and Awards}
\begin{innerlist}
\item \emph{Runner up} Best student poster at Population Genetics Group 50 \hfill 2017
\item Environment Yes! \emph{Won regional heat - runner up at the final} \hfill Sept 2016
\item EASTBIO Doctoral Training Partnership Studentship \hfill 2014-2018
\item Genetics Society, Sir Kenneth Mather Memorial Prize \hfill 2013/2014
\item University of Edinburgh, Douglas Falconer Award, best MSc dissertation \hfill 2013/2014
\item Funding for Undergraduate Summer Project:\\ Botanic Society of Scotland and the Society of Biology \hfill Summer 2012
\item \emph{Nominated}, Simon Fraser University Student Conservation Prize \hfill May 2012

\end{innerlist}


\section{Teaching}

\begin{innerlist}

\item[] Supervision \hfill\\
	Carolina Barata - Master's project - \emph{Now PhD student at the University of St. Andrews}\\
	Brice Lecher - Honour's project - \emph{Now MSc student at Université Claude Bernard}\\
\item[] Statistics and Data Analysis, MSc course \hfill 2014-2017\\
	\emph{Demonstrated in computer practical sessions, ran tutorials on probability theory and statistical analysis and marked term papers}\\
\item[] Population and Quantitative Genetics, MSc course \hfill 2015-2017 \\    
	\emph{Ran tutorial sessions on population genetic theory}\\
\item[] Ecology and Evolutionary Genetics, BSc course \hfill 2014-2015\\
	\emph{Demonstrated in computer practical sessions on evolutionary biology}



\end{innerlist}

\halfblankline


\section{Interests}

Aside from evolutionary biology I have several hobbies that I try and find time for. I enjoy playing guitar, woodworking (particularly woodturning), helping out around my parents' farm and hill-walking.


%\section{References}

%Professor Peter Keightley
%\begin{innerlist}
%\item[]  Institute of Evolutionary Biology \hfill{E-mail: %peter.keightley@ed.ac.uk}\\
%University of Edinburgh
%\end{innerlist}

%\halfblankline

%Professor Brian Charlesworth
%\begin{innerlist}
%\item[]  Institute of Evolutionary Biology \hfill{E-mail: %brian.charlesworth@ed.ac.uk}\\
%University of Edinburgh
%\end{innerlist}

\end{document}

%%%%%%%%%%%%%%%%%%%%%%%%%% End CV Document %%%%%%%%%%%%%%%%%%%%%%%%%%%%%

%----------------------------------------------------------------------%
% The following is copyright and licensing information for
% redistribution of this LaTeX source code; it also includes a liability
% statement. If this source code is not being redistributed to others,
% it may be omitted. It has no effect on the function of the above code.
%----------------------------------------------------------------------%
% Copyright (c) 2007, 2008, 2009, 2010, 2011 by Theodore P. Pavlic
%
% Unless otherwise expressly stated, this work is licensed under the
% Creative Commons Attribution-Noncommercial 3.0 United States License. To
% view a copy of this license, visit
% http://creativecommons.org/licenses/by-nc/3.0/us/ or send a letter to
% Creative Commons, 171 Second Street, Suite 300, San Francisco,
% California, 94105, USA.
%
% THE SOFTWARE IS PROVIDED "AS IS", WITHOUT WARRANTY OF ANY KIND, EXPRESS
% OR IMPLIED, INCLUDING BUT NOT LIMITED TO THE WARRANTIES OF
% MERCHANTABILITY, FITNESS FOR A PARTICULAR PURPOSE AND NONINFRINGEMENT.
% IN NO EVENT SHALL THE AUTHORS OR COPYRIGHT HOLDERS BE LIABLE FOR ANY
% CLAIM, DAMAGES OR OTHER LIABILITY, WHETHER IN AN ACTION OF CONTRACT,
% TORT OR OTHERWISE, ARISING FROM, OUT OF OR IN CONNECTION WITH THE
% SOFTWARE OR THE USE OR OTHER DEALINGS IN THE SOFTWARE.
%----------------------------------------------------------------------%
