%%%%%%%%%%%%%%%%%%%%%%%%%%%%%%%%%%%%%%%%%%%%%%%%%%%%%%%%%%%%%%%%%%%%%%%%
%%%%%%%%%%%%%%%%%%%%%% Simple LaTeX CV Template %%%%%%%%%%%%%%%%%%%%%%%%
%%%%%%%%%%%%%%%%%%%%%%%%%%%%%%%%%%%%%%%%%%%%%%%%%%%%%%%%%%%%%%%%%%%%%%%%

%%%%%%%%%%%%%%%%%%%%%%%%%%%%%%%%%%%%%%%%%%%%%%%%%%%%%%%%%%%%%%%%%%%%%%%%
%% NOTE: If you find that it says                                     %%
%%                                                                    %%
%%                           1 of ??                                  %%
%%                                                                    %%
%% at the bottom of your first page, this means that the AUX file     %%
%% was not available when you ran LaTeX on this source. Simply RERUN  %%
%% LaTeX to get the ``??'' replaced with the number of the last page  %%
%% of the document. The AUX file will be generated on the first run   %%
%% of LaTeX and used on the second run to fill in all of the          %%
%% references.                                                        %%
%%%%%%%%%%%%%%%%%%%%%%%%%%%%%%%%%%%%%%%%%%%%%%%%%%%%%%%%%%%%%%%%%%%%%%%%

%%%%%%%%%%%%%%%%%%%%%%%%%%%% Document Setup %%%%%%%%%%%%%%%%%%%%%%%%%%%%

% Don't like 10pt? Try 11pt or 12pt
\documentclass[11pt]{article}

% The automated optical recognition software used to digitize resume
% information works best with fonts that do not have serifs. This
% command uses a sans serif font throughout. Uncomment both lines (or at
% least the second) to restore a Roman font (i.e., a font with serifs).
\usepackage{times}
\renewcommand{\familydefault}{\sfdefault}

% This is a helpful package that puts math inside length specifications
\usepackage{calc}
\usepackage{comment}
\usepackage{etaremune}

\newcommand\tab[1][1cm]{\hspace*{#1}}

% Simpler bibsection for CV sections
% (thanks to natbib for inspiration)
\makeatletter
\newlength{\bibhang}
\setlength{\bibhang}{1em} %1em}
\newlength{\bibsep}
 {\@listi \global\bibsep\itemsep \global\advance\bibsep by\parsep}
\newenvironment{bibsection}%
        {\begin{etaremune}{}{%
%        {\begin{list}{}{%
       \setlength{\leftmargin}{\bibhang}%
       \setlength{\itemindent}{-\leftmargin}%
       \setlength{\itemsep}{\bibsep}%
       \setlength{\parsep}{\z@}%
        \setlength{\partopsep}{0pt}%
        \setlength{\topsep}{0pt}}}
        {\end{etaremune}\vspace{-.6\baselineskip}}
%        {\end{list}\vspace{-.6\baselineskip}}
\makeatother

% Layout: Puts the section titles on left side of page
\reversemarginpar


%% Use these lines for letter-sized paper
\usepackage[paper=letterpaper,
            %includefoot, % Uncomment to put page number above margin
            marginparwidth=1.2in,     % Length of section titles
            marginparsep=.05in,       % Space between titles and text
            margin=0.6in,               % 1 inch margins
            includemp]{geometry}

%% Use these lines for A4-sized paper
%\usepackage[paper=a4paper,
%            %includefoot, % Uncomment to put page number above margin
%            marginparwidth=30.5mm,    % Length of section titles
%            marginparsep=1.5mm,       % Space between titles and text
%            margin=25mm,              % 25mm margins
%            includemp]{geometry}

%% More layout: Get rid of indenting throughout entire document
\setlength{\parindent}{0in}

\usepackage[shortlabels]{enumitem}

%% Reference the last page in the page number
%
% NOTE: comment the +LP line and uncomment the -LP line to have page
%       numbers without the ``of ##'' last page reference)
%
% NOTE: uncomment the \pagestyle{empty} line to get rid of all page
%       numbers (make sure includefoot is commented out above)
%
\usepackage{fancyhdr,lastpage}
\pagestyle{fancy}
\pagestyle{empty}      % Uncomment this to get rid of page numbers
\fancyhf{}\renewcommand{\headrulewidth}{0pt}
\fancyfootoffset{\marginparsep+\marginparwidth}
\newlength{\footpageshift}
\setlength{\footpageshift}
          {0.5\textwidth+0.5\marginparsep+0.5\marginparwidth-2in}
\lfoot{\hspace{\footpageshift}%
       \parbox{4in}{\, \hfill %
                    \arabic{page} of \protect\pageref*{LastPage} % +LP
%                    \arabic{page}                               % -LP
                    \hfill \,}}

% Finally, give us PDF bookmarks
\usepackage{color,hyperref}
\definecolor{darkblue}{rgb}{0.0,0.0,0.3}
\hypersetup{colorlinks,breaklinks,
            linkcolor=darkblue,urlcolor=darkblue,
            anchorcolor=darkblue,citecolor=darkblue}

%%%%%%%%%%%%%%%%%%%%%%%% End Document Setup %%%%%%%%%%%%%%%%%%%%%%%%%%%%


%%%%%%%%%%%%%%%%%%%%%%%%%%% Helper Commands %%%%%%%%%%%%%%%%%%%%%%%%%%%%

% The title (name) with a horizontal rule under it
% (optional argument typesets an object right-justified across from name
%  as well)
%
% Usage: \makeheading{name}
%        OR
%        \makeheading[right_object]{name}
%
% Place at top of document. It should be the first thing.
% If ``right_object'' is provided in the square-braced optional
% argument, it will be right justified on the same line as ``name'' at
% the top of the CV. For example:
%
%       \makeheading[\emph{Curriculum vitae}]{Your Name}
%
% will put an emphasized ``Curriculum vitae'' at the top of the document
% as a title. Likewise, a picture could be included:
%
%   \makeheading[\includegraphics[height=1.5in]{my_picutre}]{Your Name}
%
% the picture will be flush right across from the name.
\newcommand{\makeheading}[2][]%
        {\hspace*{-\marginparsep minus \marginparwidth}%
         \begin{minipage}[t]{\textwidth+\marginparwidth+\marginparsep}%
             {\large \bfseries #2 \hfill #1}\\[-0.15\baselineskip]%
                 \rule{\columnwidth}{1pt}%
         \end{minipage}}

\newcommand{\makeheadingGeneric}[2][]%
        {\hspace*{-\marginparsep minus \marginparwidth}%
         \begin{minipage}[t]{\textwidth+\marginparwidth+\marginparsep}%
             {\large \bfseries #2 \hfill #1}\\[-0.15\baselineskip]%
                 \rule{\columnwidth}{1pt}%
         \end{minipage}}
% The section headings
%
% Usage: \section{section name}
\renewcommand{\section}[1]{\pagebreak[3]%
    \hyphenpenalty=10000%
    \vspace{1.3\baselineskip}%
    \phantomsection\addcontentsline{toc}{section}{#1}%
    \noindent\llap{\scshape\smash{\parbox[t]{\marginparwidth}{\raggedright #1}}}%
    \vspace{-\baselineskip}\par}

% An itemize-style list with lots of space between items
\newenvironment{outerlist}[1][\enskip\textbullet]%
        {\begin{itemize}[#1,leftmargin=*]}{\end{itemize}%
         \vspace{-.6\baselineskip}}

% An environment IDENTICAL to outerlist that has better pre-list spacing
% when used as the first thing in a \section

\newenvironment{lonelist}[1][\enskip\textbullet]%
        {\begin{list}{#1}{%
        \setlength{\partopsep}{0pt}%
        \setlength{\topsep}{0pt}}}
        {\end{list}\vspace{-.5\baselineskip}}

% An itemize-style list with little space between items
\newenvironment{innerlist}[1][\enskip\textbullet]%
        {\begin{itemize}[#1,leftmargin=*,parsep=0pt,itemsep=0pt,topsep=0pt,partopsep=0pt]}
        {\end{itemize}}

% An environment IDENTICAL to innerlist that has better pre-list spacing
% when used as the first thing in a \section
\newenvironment{loneinnerlist}[1][\enskip\textbullet]%
        {\begin{itemize}[#1,leftmargin=*,parsep=0pt,itemsep=0pt,topsep=0pt,partopsep=0pt]}
        {\end{itemize}\vspace{-.5\baselineskip}}

% To add some paragraph space between lines.
% This also tells LaTeX to preferably break a page on one of these gaps
% if there is a needed pagebreak nearby.
\newcommand{\blankline}{\quad\pagebreak[2]}
\newcommand{\halfblankline}{\quad\vspace{-0.3\baselineskip}\pagebreak[2]}

% Uses hyperref to link DOI
\newcommand\doilink[1]{\href{http://dx.doi.org/#1}{#1}}
\newcommand\doi[1]{doi:\doilink{#1}}

% For \url{SOME_URL}, links SOME_URL to the url SOME_URL
\providecommand*\url[1]{\href{#1}{#1}}
% Same as above, but pretty-prints SOME_URL in teletype fixed-width font
\renewcommand*\url[1]{\href{#1}{\texttt{#1}}}

% For \email{ADDRESS}, links ADDRESS to the url mailto:ADDRESS
\providecommand*\email[1]{\href{mailto:#1}{#1}}
% Same as above, but pretty-prints ADDRESS in teletype fixed-width font
%\renewcommand*\email[1]{\href{mailto:#1}{\texttt{#1}}}

%\providecommand\BibTeX{{\rm B\kern-.05em{\sc i\kern-.025em b}\kern-.08em
%    T\kern-.1667em\lower.7ex\hbox{E}\kern-.125emX}}
%\providecommand\BibTeX{{\rm B\kern-.05em{\sc i\kern-.025em b}\kern-.08em
%    \TeX}}
\providecommand\BibTeX{{B\kern-.05em{\sc i\kern-.025em b}\kern-.08em
    \TeX}}
\providecommand\Matlab{\textsc{Matlab}}

%%%%%%%%%%%%%%%%%%%%%%%% End Helper Commands %%%%%%%%%%%%%%%%%%%%%%%%%%%

%%%%%%%%%%%%%%%%%%%%%%%%% Begin CV Document %%%%%%%%%%%%%%%%%%%%%%%%%%%%

\begin{document}
\makeheading{Tom R. Booker}
%
%\section{Contact Information}
%
%% NOTE: Mind where the & separators and \\ breaks are in the following
%%       table.
%%
%% ALSO: \rcollength is the width of the right column of the table
%%       (adjust it to your liking; default is 1.85in).
%%
%\newlength{\rcollength}\setlength{\rcollength}{1.4in}%
%%
%\begin{tabular}[t]{@{}p{\textwidth-\rcollength}p{\rcollength}}
%t.r.booker@sms.ed.ac.uk & Tel. +447858896621 \\
%mrtombooker@gmail.com \\\\
%Institute of Evolutionary Biology\\ Ashworth Laboratories\\ Edinburgh, EH9 3FL
%\end{tabular}
%


%\section{Objective}

%Insert text here if you want to
%\begin{innerlist}
%\item More information and auxiliary documents can be found at\\\url{http://www.tedpavlic.com/facjobsearch/}
%\end{innerlist}

\section{Research Interests}

Theoretical and empirical population genetics, biodiversity, evolution, genomics, bioinformatics, statistical analysis

\section{Employment}

{\textbf{University of British Columbia}}, Vancouver, Canada

\begin{innerlist}
\item[]-	Postdoctoral Research Fellow (Sept 2018 - October 2020)
\item[]-	Supervised by Professor Michael Whitlock and Assistant Professor Samuel Yeaman (University of Calgary)
\end{innerlist}
	
\section{Education}

{\textbf{University of Edinburgh}},
Edinburgh, Scotland
\begin{outerlist}

\item[] PhD.,
             \href{http://www.ed.ac.uk/biology/evolutionary-biology} {Evolutionary Genetics},
             October 2014 - October 2018  
        \begin{innerlist}
        \item[]- Thesis Title: \textsc{Understanding patterns of genetic diversity in the house mouse genome}
        \item[]- Supervisors: Professor Peter Keightley and Professor Brian Charlesworth

        \end{innerlist}

\item[] MSc.,
        \href{http://qgen.bio.ed.ac.uk/}
             {Evolutionary Genetics},
             2013 - 2014 (Distinction)
        \begin{innerlist}
        \item[]- Thesis Title: \textsc{Searching for balancing selection on a mimicry supergene in the Batesian mimic} \textit{Papilio polytes}
        \item[]- Supervisors: Professor Deborah Charlesworth and Assistant Professor Rob W. Ness

        \end{innerlist}
\end{outerlist}
\vspace{.1in}
\href{}{\textbf{University of Stirling}},
Stirling, Scotland
\begin{outerlist}
\item[] BSc Hons,
        \href{http://www.stir.ac.uk/natural-sciences/about-us/bes/}
             {Ecology}, 2009 - 2013 (First Class)
        \begin{innerlist}
        \item[]- Dissertation Title: \textsc{An investigation into the fitness and distribution of a newly discovered allopolyploid species,} \textit{Mimulus peregrinus}
        \item[]- Supervisor:
                   Dr Mario Vallejo-Marin
		\item[]- Study abroad at Simon Fraser University, Vancouver, Canada. 2011-2012.
        \end{innerlist}
\end{outerlist}



\section{Papers}
\vspace{-.1275in}
\begin{bibsection}


    \item {\bf Booker, T. R.}, Yeaman S. \& Whitlock M. C. (\emph{In revision - Evolution Letters}). ``Global adaptation confounds the search for local adaptation". \\
     Preprint available at: https://www.biorxiv.org/content/10.1101/742247v1


	\item Byers K.A., {\bf Booker T. R.}, Combs M., Himsworth C.G., Munshi-South J., Patrick D.M., Whitlock M.C.. (\emph{Accepted})  ``Using genetic relatedness to understand heterogeneous distributions of urban rat-associated pathogens``. \emph{Evolutionary Applications}

    \item {\bf Booker, T. R.}, Yeaman S. \& Whitlock M. C. (2020) ``Variation in recombination rate affects detection of $F_{ST}$ outliers under neutrality".\\
	\emph{Molecular Ecology}, Early Access - \emph{Highlighted on the cover and accompanying News and Views piece}
   
    \item {\bf Booker, T. R.} (2020) ``Inferring parameters of the distribution of fitness effects of new mutations when beneficial mutations are strongly advantageous and rare". \\
     G3: GENES, GENOMES, GENETICS Early online May 5, 2020;

    \item {\bf Booker, T. R.}, \& Keightley, P. D. (2018). ``Understanding the factors that shape patterns of nucleotide diversity in the house mouse genome". \emph{Molecular Biology and Evolution}, 35(12) 2971-2988
    
   \item {\bf Booker, T. R.}, Jackson, B. C., \& Keightley, P. D. (2017). ``Detecting positive selection in the genome". \emph{BMC Biology}, 15:98. 

	\item {\bf Booker, T. R.}, Ness, R. W., \& Keightley, P. D. (2017). ``The recombination landscape in wild house mice inferred using population genomic data". \emph{Genetics}, 207(1) 297-309

	\item Keightley, P. D., Campos, J. L., {\bf Booker, T. R.}, \& Charlesworth, B. (2016). ``Inferring the frequency spectrum of derived variants to quantify adaptive molecular evolution in protein-coding genes of \textit{Drosophila melanogaster}". \emph{Genetics}, 203(2), 975-984.
	
	\item {\bf Booker, T.}, Ness, R. W., \& Charlesworth, D. (2015). ``Molecular evolution: breakthroughs and mysteries in Batesian mimicry". \emph{Current Biology}, 25(12), R506-R508.

\end{bibsection}

 \section{Papers in preparation}
\vspace{-.1275in}
\begin{itemize}

    
    \item {\bf Booker, T. R.}, Jackson, B. Craig, R. Charlesworth, B. \& Keightley, P. D.  ``Patterns of genetic diversity around protein-coding
exons and conserved non-coding elements are explained by strong selective sweeps in mice".

\end{itemize}
	
\section{Academic Honours and Awards}
\begin{innerlist}
\item Registration Award - Society of Molecular Biology and Evolution \hfill 2019
\item \emph{Runner up} Best student talk at Population Genetics Group 51 \hfill 2018
\item \emph{Runner up} Best student poster at Population Genetics Group 50 \hfill 2017
\item Environment Yes! \emph{Won regional heat - runner up at the national final} \hfill Sept 2016
\item EASTBIO Doctoral Training Partnership Studentship \hfill 2014-2018
\item Genetics Society, Sir Kenneth Mather Memorial Prize \hfill 2013/2014
\item University of Edinburgh, Douglas Falconer Award, best MSc dissertation \hfill 2013/2014
\item Funding for Undergraduate Summer Project:\\ Botanic Society of Scotland and the Society of Biology \hfill Summer 2012
\item \emph{Nominated}, Simon Fraser University Student Conservation Prize \hfill May 2012

\end{innerlist}


\section{Academic Service}
\begin{innerlist}
	\item[] I have reviewed articles for the following journals:
		\begin{innerlist}
			\item[] \emph{PLoS Genetics}, \emph{Molecular Biology and Evolution}, \emph{Genome Biology and Evolution}, \emph{Ecology and Evolution}, \emph{Frontiers in Zoology}
		\end{innerlist}

\halfblankline

	\item 2019 - Poster Judge BIOL 310 Animal Behaviour - UBC 

	\item 2019 - \emph{Ongoing} Co-organiser of the Vancouver Evolution Group (VEG)

	\item 2019 I took part in a mentor scheme for undergraduate students attending SMBE 2019 in Manchester, UK

	\item 2017 I started and organised a journal club on classic population genetic papers at the University of Edinburgh in 2017




\end{innerlist}

\blankline


\section{Teaching}

{\bf Mentoring} 

\tab C. Atkinson -  Directed studies co-supervisor - \emph{Undergraduate student at UBC}\\

\tab K.A. Byers -  Bioinformatics/genomics mentor - \emph{PhD student at UBC}\\

\tab S-A. Xerri - Master's project co-supervisor - \emph{Now PhD student at the Max Planck Institute}\\

\tab C. Barata - Master's project co-supervisor- \emph{Now PhD student at the University of St. Andrews}\\

\tab B. Lecher - Honour's project co-supervisor- \emph{Now Pre Doctoral Fellow at the European Bioinformatics Institute} \\

{\bf Course Instruction}
\begin{innerlist}
\item[] Statistics and Data Analysis, MSc course \hfill 2014-2017\\
	\emph{Demonstrated in computer practical sessions, ran tutorials on probability theory and statistical analysis and marked term papers}\\
\item[] Population and Quantitative Genetics, MSc course \hfill 2015-2017 \\    
	\emph{Ran tutorial sessions on population genetic theory}\\
\item[] Ecology and Evolutionary Genetics, BSc course \hfill 2014-2015\\
	\emph{Demonstrated in computer practical sessions on evolutionary biology}
\end{innerlist}

\halfblankline

\section{Selected Presentations}

{\bf January 2020} - American Society of Naturalists 2020, Asilomar, USA (Talk) \\
\emph{Global adaptation confounds the search for local adaptation} \\

{\bf October 2019}  - EcoEvo Retreat, Squamish, Canada (Talk) \\
\emph{Leveraging linkage information in studies of local adaptation} \\

{\bf September 2019} - BLISS, UBC, Vancouver, Canada (Talk) \\
\emph{Global adaptation confounds the search for local adaptation} \\

{\bf July 2019} - SMBE, Manchester, UK (Poster) \\
\emph{Patterns of genetic diversity around protein-coding exons and conserved non-coding elements are explained by strong selective sweeps in mice} \\

{\bf September 2018} - EcoEvo Retreat, Squamish, Canada (Talk) \\
\emph{Estimating the parameters of selective sweeps from patterns of diversity around functional elements in wild house mice} Mus musculus castaneus\\

{\bf January 2018} - Population Genetics Group 51, Bristol, UK (Talk) \\
\emph{Estimating the parameters of selective sweeps from patterns of diversity around functional elements in wild house mice} Mus musculus castaneus\\

{\bf August 2017} - ESEB 2017, Groningen, Netherlands (Poster) \\
\emph{Selective sweeps and background selection in the genome of wild house mice,} Mus musculus castaneus \\

{\bf January 2017} - Population Genetics Group 50, 2017, Cambridge, UK (Poster) \\
\emph{Selective sweeps and background selection in the genome of wild house mice,} Mus musculus castaneus \\

{\bf July 2016} - SMBE, Gold Coast, Australia (Talk) \\
\emph{Hill-Robertson Interference in wild mice,} Mus musculus castaneus\\

{\bf December 2015} - Population Genetics Group 49, Edinburgh, UK (Talk - Invited) \\
\emph{Hill-Robertson Interference in wild mice,} Mus musculus castaneus\\


{\bf July 2015} - SMBE, 2015, Vienna, Austria (Poster) \\
\emph{Selective sweeps and background selection in the genome of wild house mice,} Mus musculus castaneus \\

{\bf May 2015} - Quantitative Genomics, 2015, London, UK (Talk) \\
\emph{Simulating genome evolution in the house mouse: understanding the contribution of Hill-Robertson interference to patterns of genetic diversity} \\


\pagebreak

\makeheadingGeneric{References}

\blankline

Professor Michael C. Whitlock
\begin{innerlist}
\item[] Postdoctoral Supervisor
\item[] Department of Zoology \hfill{E-mail: whitlock@zoology.ubc.ca}\\
University of British Columbia
\end{innerlist}

\halfblankline

Professor Peter Keightley
\begin{innerlist}
\item[] PhD Supervisor
\item[]  Institute of Evolutionary Biology \hfill{E-mail: peter.keightley@ed.ac.uk}\\
University of Edinburgh
\end{innerlist}

\halfblankline

Professor Deborah Charlesworth
\begin{innerlist}
\item[] MSc Supervisor
\item[]  Institute of Evolutionary Biology \hfill{E-mail: deborah.charlesworth@ed.ac.uk}\\
University of Edinburgh
\end{innerlist}


\end{document}

	




%%%%%%%%%%%%%%%%%%%%%%%%%% End Document %%%%%%%%%%%%%%%%%%%%%%%%%%%%%
